\hypertarget{index_Ipass}{}\section{Ipass by Olivier Verwoerd}\label{index_Ipass}
This is the documentation of the files I made for the Ipass of the university of applied sciances Utrecht. it\textquotesingle{}s the curiosity of useing analog and different signals that is not perfectly made to be understoud by the arduino. Useing morse code was a perfect way to do this since morse is not perfecty timed.\hypertarget{index_work}{}\section{What did I do}\label{index_work}

\begin{DoxyEnumerate}
\item First I made it posible to get morse form a digital signal.
\item Than i made an dictionary to get char form morse
\item I wanted to make it faster and faster so I optimalized a few things
\item Wanted even faster than sound and i found out morse is still used (sort of) to send bytes in IR remotes and made support for that.
\item Everything above was digital so i wanted to do something analog. To start with volume measureing.
\item than frequency detection.
\item I\textquotesingle{}ve made some configuration tools in the code to debug and make the script more dynamic
\item I wanted to make use of the frequency to filter and the volume detection the morse code to make it even better. This didn\textquotesingle{}t work out. Frequency detection only worked with no background noise because it couldn\textquotesingle{}t detect multiple frequencys. It\textquotesingle{}s not a good spectrometer. This just made the morse detection significant more unreliable. Volume is more unreliable with a lower buffer size and therefore didn\textquotesingle{}t work, volume is too slow with a big buffer. This isn\textquotesingle{}t inposible to implement. I never just got the configuration right and decided to let the hardware do it\textquotesingle{}s work.
\item Well. You are looking at my last thing i did. Doxygen documentation...
\end{DoxyEnumerate}\hypertarget{index_handy}{}\section{Good to know}\label{index_handy}
N\+O\+TE\+: Since printing thousands of numbers in an array is simply too much to make any sense to me. i\textquotesingle{}ve used this tool \href{https://developer.mbed.org/users/borislav/notebook/serial-port-plotter/}{\tt https\+://developer.\+mbed.\+org/users/borislav/notebook/serial-\/port-\/plotter/} A great tool for plotting the array. To let the tool work there needs to be an \$ in front of the number followed by an ;

Because i\textquotesingle{}m inpactient to do wait for the slow terminal. I used baud rate of 19200 A\+ND N\+OT 2400. Change this in makefile.\+due and in hwlib\+\_\+ostream change B\+M\+P\+T\+K\+\_\+\+B\+A\+U\+D\+R\+A\+TE to 19200.\hypertarget{index_licence}{}\section{licence}\label{index_licence}
Copyright Olivier Verwoerd 2017. Distributed under the Boost Software License, Version 1.\+0. (See accompanying file L\+I\+C\+E\+N\+S\+E\+\_\+1\+\_\+0.\+txt or copy at \href{http://www.boost.org/LICENSE_1_0.txt}{\tt http\+://www.\+boost.\+org/\+L\+I\+C\+E\+N\+S\+E\+\_\+1\+\_\+0.\+txt})\hypertarget{index_dependency}{}\section{dependency}\label{index_dependency}
It needs this \href{https://github.com/wovo/hwlib}{\tt https\+://github.\+com/wovo/hwlib} and some love otherwise this will never work. 